\section{Introducción}

\begin{frame}{Análisis de salida}
    \begin{itemize}
        \item El \textbf{Análisis de Salida} es la evaluación de los resultados generados por un modelo de simulación, con el propósito de predecir el desempeño absoluto del sistema a través de las medidas de desempeño \cite{BCN}.
    \end{itemize}
\end{frame}

\begin{frame}{Análisis de salida}
    \begin{itemize}
        \item Como las entradas del modelo son aleatorias, sus salidas también lo serán (\textit{RIRO}) y se requiere un tratamiento estadístico de los resultados.
        \item Los resultados son estimadores de las medidas de desempeño reales (desconocidas), una realización particular de variables aleatorias que pueden tener grandes varianzas \cite{LK}.
        \item Dependen de factores como:
        \begin{itemize}
            \item Parámetros de entrada (factores controlables y no controlables).
            \item Condiciones iniciales del modelo.
        \end{itemize}
    \end{itemize}
\end{frame}

\begin{frame}{Análisis de salida}
    \begin{itemize}
        \item Las principales actividades que se desarrollan durante el análisis de salida son:
        \begin{enumerate}
            \item Determinación del número de réplicas.
            \item Determinación de la longitud de corrida y del periodo de calentamiento.
            \item Estimación de las medidas de desempeño.
        \end{enumerate}
    \end{itemize}
\end{frame}

\begin{frame}{Análisis de salida}
    \begin{itemize}
        \item Hay varias situaciones a considerar. Entre ellas están:
        \begin{itemize}
            \item El tipo de simulación con respecto al horizonte de tiempo de la simulación.
            \item El tipo de medida de desempeño a analizar.
            \item La influencia de las condiciones iniciales.
            \item Posible autocorrelación en los datos.
        \end{itemize}
    \end{itemize}
\end{frame}