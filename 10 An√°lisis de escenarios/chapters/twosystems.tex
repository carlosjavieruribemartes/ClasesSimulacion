\section{Comparación de dos sistemas}
 %Comparing two systems
\begin{frame}{Comparación de dos sistemas}
    \begin{itemize}
        \item Suponga que se tienen muestras de dos poblaciones (configuraciones del sistema): $X_{11}, X_{12},\dots,X_{1n_1}$ una muestra de tamaño $n_1$ de la configuración del sistema 1 y $X_{21}, X_{22},\dots,X_{2n_2}$ una muestra de tamaño $n_2$ de la configuración del sistema 2, las cuales representan una medida de desempeño que será utilizada para determinar cuál de los dos sistemas es preferido.
        \item Asuma que cada configuración tiene una media poblacional desconocida para la medida de desempeño, $E[X_1]=\theta_1$ y $E[X_2]=\theta_2$.

    \end{itemize}
\end{frame}

\begin{frame}{Comparación de dos sistemas}
    \begin{itemize}
        \item El problema es determinar, con alguna confianza estadística, si $\theta_1<\theta_2$ o si $\theta_1>\theta_2$. 
        \item Si se define $\theta=\theta_1-\theta_2$ como la diferencia promedio en el desempeño de los dos sistemas, basta con determinar si $\theta>0$ o $\theta<0$, para determinar si $\theta_1<\theta_2$ o si $\theta_1>\theta_2$.
        \item Es decir, es suficiente concentrarse en la diferencia del desempeño de los sistemas más que en los valores de cada uno.
    \end{itemize}
\end{frame} 

\begin{frame}{Comparación de dos sistemas}
    \begin{itemize}
        \item Calcule un intervalo de confianza (IC) para $\theta=\theta_1-\theta_2$.
        \item Si IC está totalmente a la izquierda de 0, entonces hay evidencia estadística fuerte de que $\theta_1-\theta_2<0$ o $\theta_1 < \theta_2$.
        \item Si IC está totalmente a la derecha de 0, entonces hay evidencia estadística fuerte de que $\theta_1-\theta_2>0$ o $\theta_1 > \theta_2$.
        \item Si IC contiene al 0, entonces no hay evidencia estadística de que alguno de los sistemas sea mejor que el otro.
    \end{itemize}
\end{frame}

\begin{frame}{Muestreo Independiente}
    \begin{itemize}
        \item El muestre independientes implica que secuencias diferentes e independientes de númeroes aleatorios serán utilizadas para simular los dos sistemas.
        \item Esto implica que todas las observaciones del sistema 1 simulado son estadísticamente independientes de las observaciones del sistema 2 simulado. 
    \end{itemize}
\end{frame}

\begin{frame}{Muestreo Independiente}
    \begin{itemize}
        \item Sea $\bar{X_1}$, $\bar{X_2}$, $s_1^2$ y $s_2^2$ las medias muestrales y varianzas muestrales para las dos muestras ($k=1,2$). Entonces:
        \[\bar{X_k}=\frac{1}{n_k}\sum_{j=1}^{n_k}{X_{kj}}\]
        \[s_k^2=\frac{1}{n_k-1}\sum_{j=1}^{n_k}{(X_{kj}-\bar{X_k})^2}\]
    \end{itemize}
\end{frame}

\begin{frame}{Muestreo Independiente}
    \begin{itemize}
        \item Se requiere un estimado de $\theta=\theta_1-\theta_2$, el cual puede ser estimado por la diferencia $\hat{D}=\bar{X_1}-\bar{X_2}$.
        \item Asumiendo de las muestras son independientes, la varianza de la diferencia es
        \[Var(\hat{D})=Var(\bar{X_1}-\bar{X_2})=\frac{\sigma_1^2}{n_1}+\frac{\sigma_2^2}{n_2}\]
        donde $\sigma_1^2$ y $\sigma_2^2$ son las varianzas poblacionales (desconocidas).
    \end{itemize}
\end{frame}

\begin{frame}{Muestreo Independiente}
    \begin{itemize}
        \item Si se puede asumir que ambas poblaciones tienen una varianza común $\sigma^2=\sigma_1^2=\sigma_2^2$, un estimador de $\sigma^2$ se puede definir como:
        \[s_p^2=\frac{(n_1-1)s_1^2+(n_2-1)s_2^2}{n_1+n_2-2}\]
        \item Entonces, un intervalo de confianza $(1-\alpha)\%$ para $\theta$ es:
        \[\hat{D}\pm t_{\alpha/2,v}s_p\sqrt{\frac{1}{n_1}+\frac{1}{n_2}}\]
        donde $v=n_1+n_2-2$
    \end{itemize}
\end{frame}

\begin{frame}{Muestreo Independiente}
    \begin{itemize}
        \item Para el caso de varianzas diferentes, un intervalo de confianza $(1-\alpha)\%$ para $\theta$ es:
        \[\hat{D}\pm t_{\alpha/2,v}\sqrt{\frac{s_1^2}{n_1}+\frac{s_2^2}{n_2}}\]
        donde
        \[v=\left \lfloor{\frac{(s_1^2/n_1+s_2^2/n_2)^2}{\frac{(s_1^2/n_1)^2}{n_1+1}+\frac{(s_2^2/n_2)^2}{n_2+1}}-2}\right \rfloor\]
    \end{itemize}
\end{frame}

\begin{frame}{Muestreo correlacionado}
    \begin{itemize}
        \item También conocido como CRN (por las siglas en inglés para \textit{Common Random Numbers}), implica que para cada réplica, los mismos números aleatorios son utilizados para simular ambos sistemas.
        \item Se continua asumiendo que las observaciones dentro de una muestra son independientes e idénticamente distribuidas; sin embargo, las muestras no son independientes.
    \end{itemize}
\end{frame}

\begin{frame}{Muestreo correlacionado}
    \begin{itemize}
        \item Se asume que se tiene igual número de réplicas para ambos sistemas, esto es, $n_1=n_2=n$.
        \item Como cada réplica para los dos sistemas utiliza las mismas secuencias de números aleatorios, la correlación entre $\left(X_{1j},X_{2j}\right)$ será diferente de cero.
        \item Aún así, cada par será independiente \textit{entre} réplicas.
    \end{itemize}
\end{frame}

\begin{frame}{Muestreo correlacionado}
    \begin{itemize}
        \item Para analizar este caso se calculan las diferencias de cada par:
        \[D_j=X_{1j}-X_{2j} \quad\text{ para } j=1,2,\dots, n\]
        \item Entonces, $D_1, D_2, \dots, D_n$, forman una muestra aleatoria, que puede ser analizada por los métodos tradicionales.
    \end{itemize}
\end{frame}

\begin{frame}{Muestreo correlacionado}
    \begin{itemize}
        \item Un intervalo de confianza para $\theta=\theta_1-\theta_2$, con una significancia $\alpha$, es:
        \[\hat{ D }=\frac{1}{n}\sum_{j=1}^{n}{D_j}\]
        \[s_D^2=\frac{1}{n-1} \sum_{j=1}^{n}{\left(D_j-\hat{ D} \right)^2}\]
        \[\hat{ D }\pm t_{\alpha/2,n-1}\frac{s_D}{\sqrt{n}}\]
        \item La interpretación del intervalo de confianza resultante es la misma que en el caso del muestreo independiente.
    \end{itemize}
\end{frame}