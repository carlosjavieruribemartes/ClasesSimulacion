\section{Modelado de datos}

\begin{frame}{Modelado de datos}
    \begin{itemize}
        \item En esta etapa, un modelo probabilístico es ajustado a las series de tiempo empíricas recolectadas.
        \item Dependiendo del tipo de datos de series de tiempo que se van a modelar, esta etapa se puede clasificar en dos categorías:
        \begin{enumerate}
            \item Las observaciones independientes se modelan como una secuencia de variables aleatorias iid. En este caso, se busca identificar (ajustar) una distribución y sus parámetros a los datos empíricos.
            \item Las observaciones dependientes se modelan como procesos aleatorios con dependencia temporal. En este caso, se requiere identificar (ajustar) una ley de probabilidad a los datos empíricos.
        \end{enumerate}
    \end{itemize}
\end{frame}

\begin{frame}{Identificación de distribuciones de probabilidad}
    \begin{itemize}
        \item Existen literalmente cientos de distribuciones de probabilidad.
        \item Algunas distribuciones aparecen muy a menudo en estudios de simulación:
        \begin{itemize}
            \item Binomial, Poisson, Normal, Lognormal, Exponencial, Gamma, Beta, Erlang, Weibull, Uniforme, Triangular, ...
        \end{itemize}
    \end{itemize}
\end{frame}

