\section{Simulación manual}

\begin{frame}{Simulación de eventos discretos}
    \begin{itemize}
        \item Aunque la simulación por eventos discretos puede conceptualmente realizarse a mano, la cantidad de datos que deben almacenarse y manipularse para la mayoría de las aplicaciones reales involucra el uso de un computador.
    \end{itemize}
\end{frame}

\begin{frame}{Simulación manual}{Reloj de la simulación}
    \begin{itemize}
        \item Dada la naturaleza dinámica de los modelos de simulación por eventos discretos, se requiere hacer seguimiento del valor actual del \textit{tiempo simulado}.
        \item De igual forma, se requiere un mecanismo para avanzar el tiempo simulado de un valor a otro.
        \item La variable dentro de un modelo de simulación que guarda el valor actual del tiempo simulado se llama \textit{reloj de la simulación}.
    \end{itemize}
\end{frame}

\begin{frame}{¿Cómo empezar a simular?}
    \begin{itemize}
        \item Hasta que se encuentre cómodo con sus competencias de modelado se recomienda responder a las siguientes preguntas:
        \begin{itemize}
            \item ¿Cuál es el sistema? ¿Qué información se conoce del sistema?
            \item ¿Cuáles son las medidas de desempeños requeridas?
            \item ¿Cuáles son las entidades? ¿Qué información debe ser almacenada o tenida en cuenta para cada entidad? ¿Cómo ingresas las entidades al sistema?
            \item ¿Cuáles son los recursos que utilizan las entidades? ¿Qué entidades usan cuáles recursos y cómo?
            \item ¿Cuáles son los flujos del proceso? Diagrame el flujo del proceso o un diagrama de actividad preliminar.
            \item Desarrolle un pseudo-código para el modelo o un modelo conceptual completo.
        \end{itemize}
    \end{itemize}
\end{frame}

\begin{frame}{Simulación manual}
    \begin{itemize}
        \item El analista debe definir:
        \begin{enumerate}
            \item Entradas: Parámetros exógenos que, usualmente, son independientes de otras características del sistema.
            \item Salidas: Valores utilizados para calcular las medidas de desempeño del sistema (respuestas o indicadores).
            \item Estados del sistema: Conjunto de variables de estado que definen el sistema en un momento dado.
        \end{enumerate}
    \end{itemize}
\end{frame}

\begin{frame}{Pasos para realizar una simulación manual}
    \begin{itemize}
        \item Se recomienda al analista seguir estas pautas:
        \begin{enumerate}
            \item Determine las características de cada entrada.
            \item Determine las actividades, eventos y estados del sistema relevantes.
            \item Determine el resumen de las medidas de desempeño requeridas.
            \item Determinar las salidas requeridas para calcular las medidas de desempeño.
            \item Construir una tabla de simulación.
            \item En cada paso, genere un valor para las actividades, encuentre los estados del sistema y calcule las salidas.
            \item Cuando termine la simulación, utilice las salidas para calcular las medidas de desempeño.
        \end{enumerate}
    \end{itemize}
\end{frame}

\begin{frame}{Tablas de simulación}
    \begin{itemize}
        \item Se diseña de forma tal que cada paso dependa únicamente de entradas del modelo o de uno o varios pasos o valores previamente computados.
    \end{itemize}
\end{frame}

\begin{frame}{Tablas de simulación}{Columnas}
    \begin{itemize}
        \item Cada columna puede contener:
        \begin{enumerate}
            \item Una actividad asociada con una entrada del modelo.
            \item Una variable aleatoria definida como una entrada del modelo.
            \item Un estado del sistema.
            \item Un evento, o la hora del reloj de un evento.
            \item Una salida del modelo.
            \item Una respuesta o indicador.
        \end{enumerate}
    \end{itemize}
\end{frame}

\begin{frame}{Tablas de simulación}{Filas}
    \begin{itemize}
        \item Cada fila puede representar:
        \begin{enumerate}
            \item La ocurrencia de uno o más eventos.
            \item El progreso de una entidad a través del sistema.
        \end{enumerate}
    \end{itemize}
\end{frame}

\begin{frame}{Uso de aleatorios}
    \begin{itemize}
        \item Es una buena práctica utilizar una secuencia de aleatorios y continuar de una manera sistemática, sin utilizar más de una vez la misma secuencia en un problema dado.
        \item Si la misma secuencia es usada de forma repetida, pueden ocurrir sesgos estadísticos u otros efectos no deseados que afecten los resultados.
    \end{itemize}
\end{frame}