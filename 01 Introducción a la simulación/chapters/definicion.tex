\section{Simulación}

\begin{frame}{Simulación}
    \begin{itemize}
        \item \textit{Simulación} se refiere a un conjunto de métodos y aplicaciones que buscan \textit{imitar} la operación de un sistema o proceso \cite{BCN,KSS}.
    \end{itemize}
\end{frame}

\begin{frame}{Simulación}
    \begin{itemize}
        \item Implica la generación de una \textit{historia artificial} del sistema \cite{BCN}.
        \item Y también la observación de dicha historia artificial para obtener inferencias relacionadas con las características operacionales del sistema \cite{BCN}.
    \end{itemize}
\end{frame}

\begin{frame}{Ventajas de la simulación}
    Dentro de las ventajas del uso de la simulación se pueden mencionar: 
    \begin{enumerate}
        \item Se puede experimentar con una variedad de escenarios, lo cual no sería posible en el sistema real \cite{KSS}.
        \item Se puede obtener una visión de la relación entre las variables y su importancia en el desempeño del sistema  \cite{BCN}.
        \item Se pueden encontrar cuellos de botellas y determinar en qué parte del proceso ocurren demoras excesivas  \cite{BCN}.
    \end{enumerate}
\end{frame}

\begin{frame}{Desventajas de la simulación}
    Algunas de las desventajas del uso de la simulación son \cite{BCN}: 
    \begin{enumerate}
        \item La construcción de modelos es un arte que requiere entrenamiento y que se aprende con la experiencia.
        \item Los resultados de los modelos de experimentación pueden ser difíciles de interpretar.
        \item Modelar y analizar un problema con simulación puede ser costoso y laborioso.
    \end{enumerate}
\end{frame}