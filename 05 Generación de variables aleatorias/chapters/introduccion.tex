%\section{Introducción}

\begin{frame}{Generación de variables aleatorias}
    \begin{itemize}
        \item Los modelos de simulación usualmente contienen actividades o atributos cuya duración o valor es impredecible o incierto.
        \item Para modelar este tipo de situaciones es útil emplear \textit{variables aleatorias} con distribuciones de probabilidad específicas.
        %\item Las variables aleatorias se pueden clasificar en variables aleatorias discretas y continuas.
        \item Por \textit{generar una variable aleatoria} se entiende obtener una observación de una variable aleatoria de una distribución deseada \cite{LK}.
    \end{itemize}
\end{frame}

%\begin{frame}{Generación de variables aleatorias}
%    \begin{itemize}
        
%        \item El ingrediente principal de los métodos para generar variables aleatorias de cualquier distribución o proceso aleatorio es una fuente de números aleatorios IID $U\left[0,1\right]$ \cite{LK}.
%        \item Usualmente hay varios algoritmos alternativos disponibles para generar variables aleatorias a partir de una distribución dada. Varios factores deben considerarse para escoger cuál algoritmo utilizar en un estudio en particular \cite{LK}.
%    \end{itemize}    
%\end{frame}
