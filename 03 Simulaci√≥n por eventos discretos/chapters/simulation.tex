\section{Métodos de simulación}

\begin{frame}{Introducción}
    \begin{itemize}
        \item Simulación se refiere a un conjunto de métodos y aplicaciones que buscan \textit{imitar} la operación de un sistema o proceso \cite{BCN,KSS}.
    \end{itemize}
\end{frame}

\begin{frame}{Simulación de Monte Carlo - MCS}
    \begin{itemize}
        \item Los sistemas se simulan utilizando números aleatorios para abordar problemas matemáticos que no pueden resolverse exactamente mediante el uso de técnicas analíticas o ecuaciones matemáticas.
    \end{itemize}
\end{frame}

\begin{frame}{Simulación por eventos discretos - DES}
    \begin{itemize}
        \item La operación de un sistema se representa como una secuencia cronológica de eventos para describir los flujos de personas o material y explorar los efectos de cualquier cambio.
        \item La simulación de eventos discretos es más adecuada para analizar sistemas que se pueden modelar como una serie de colas y actividades.
    \end{itemize}
\end{frame}

\begin{frame}{Modelado basado en Agentes - ABM}
    \begin{itemize}
        \item El sistema se modela como una colección de entidades autónomas de toma de decisiones, conocidas como \textit{agentes}, con el fin de explicar y comprender los complejos patrones de comportamiento emergente y la dinámica del sistema del mundo real.
    \end{itemize}
\end{frame}

\begin{frame}{Dinámica de sistemas - SD}
    \begin{itemize}
        \item Los sistemas complejos se modelan utilizando diagramas de causales con ciclos de retroalimentación y diagramas de flujo de stock para explorar su comportamiento dinámico
    \end{itemize}
\end{frame}

\begin{frame}{Simulación híbrida - HS}
    \begin{itemize}
        \item Combina dos o más técnicas de simulación para mitigar las desventajas de las técnicas individuales.
        \item En ocasiones se combina técnicas básicas de simulación con otras técnicas, como la lógica difusa (FL) y las redes neuronales (NN).
    \end{itemize}
\end{frame}